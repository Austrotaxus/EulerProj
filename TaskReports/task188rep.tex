
\documentclass[11pt,a4paper]{article}
\usepackage{listings}
\usepackage{hyperref}
\usepackage[utf8x]{inputenc}
\usepackage[russian]{babel}
\usepackage{dirtytalk}

\title{Project Euler Task 188}
\author{Козиний Сергей}
\begin{document}    
\maketitle
\href{''https://projecteuler.net/problem=188''}{Problem 188}

\say{The hyperexponentiation or tetration of a number $a$ by a positive integer $b$, denoted by $a ↑↑ b$ or $^ba$, is recursively defined by:

$a ↑↑ 1 = a$,
$a ↑↑ (k+1) = a^{(a ↑↑ k)}$.

Thus we have e.g. ${3↑↑2 = 3^3 = 27}$ , hence $3 ↑↑ 3 = 3^{27} = 7625597484987$  and  ${3 ↑↑ 4}$ is roughly $(10^{3.6383346400240996*10^{12}})$.

Find the last 8 digits of $1777 ↑↑ 1855$.}\\



Главная, и единственная проблема возникающая при решении Problem 188 - проблема памяти. Размещение в памяти чисел  ${3 ↑↑ 5}$ и больших представляется невозможным. Но поскольку нам необходимо получить только 8 последних цифр числа $1777 ↑↑ 1855$, то для решения проблемы, можно обратиться к свойствам операции $mod$:\\

$ (a+b) \quad mod \quad c = ( (a \quad mod \quad c) + (b \quad mod \quad c) )\quad mod \quad c $

$ (a*b) \quad  mod \quad c =  (a \quad mod \quad c) * (b \quad mod \quad c) \quad mod \quad c $

$ a^b \quad mod \quad c $ =  $ (a \quad mod \quad  c)^b  \quad mod \quad c $ \\

Определим операцию $hyperExpMod$ как рекурсивную функцию трёх аргументов(так как она была дана в тексте задания)

\begin{lstlisting}[language=Haskell]
--Haskell lang
hyperExpMod base 0 m = 1
hyperExpMod base pow m =
     expMod base (hyperExpMod base (pow-1) m) m
\end{lstlisting}

Вспомогательная операция ${expMod}$ решает проблему получения остатка от деления $a^b$ на $m$ и использует модифицированный алгоритм бинарного (быстрого) возведения в степень.
Модификация заключается в том, что на каждой итерации $expMod$ от основания $b$ и аккумулятора $p$ берётся остаток от деления на $m$(что имеем право делать по свойствам операции $mod$). Этим гарантирует то, что при рекурсивном вызове $expMod$, на каждой итерации, $b$ и $p$ будут меньше чем $m$, a следовательно уместятся в память.

\begin{lstlisting}[language=Haskell]
-- Haskell lang
expMod b p m = expModIter 1 b p m
            expModIter res b p m
                | p==0 = res `mod` m
                | p `mod` 2 == 0 = expModIter res ((b^2)`mod`m) (div p 2) m
                | otherwise = expModIter (res*b `mod` m) (b`mod`m) (p-1) m
\end{lstlisting}

Поскольку нам необходимо найти последние восемь цифр данного числа целесоообразно воспользоваться типом $Integer$ который обеспечивает работу с длинной арифметикой (на каждой из итераций $expMod$, $mod \quad m$ будет вычисляться от произведения чисел каждое из которых $\le  10^8$, а значит левый аргумент $mod$ будет $\le  10^{16}$ что может не уместиться в обычныйй $Int$ ). Для этого достаточно указать сигнатуру типа для результирующего значения
\begin{lstlisting}[language=Haskell]
  -- Haskell lang
  task188::Integer
  task188 = hyperExpMod 1777 1855 100000000
          where
            hyperExpMod base 0 m = 1
            hyperExpMod base pow m =
                expMod base (hyperExpMod base (pow-1) m) m
            expMod b p m = expModIter 1 b p m
            expModIter res b p m
                | p==0 = res `mod` m
                | p `mod` 2 == 0 = expModIter res ((b^2)`mod`m) (div p 2) m
                | otherwise = expModIter (res*b `mod` m) (b`mod`m) (p-1) m

\end{lstlisting}
Результат:
\begin{lstlisting}[language=Bash]
#Console
austrotaxus@small-box:~/EulerProj$ stack ghci
Using main module: Package `EulerProj' component exe:EulerProj-exe with main-is file: /home/austrotaxus/EulerProj/app/Main.hs
The following GHC options are incompatible with GHCi and have not been passed to it: -threaded
Configuring GHCi with the following packages: EulerProj
GHCi, version 7.10.3: http://www.haskell.org/ghc/  :? for help
[1 of 2] Compiling Lib              ( /home/austrotaxus/EulerProj/src/Lib.hs, interpreted )
[2 of 2] Compiling Main             ( /home/austrotaxus/EulerProj/app/Main.hs, interpreted )
Ok, modules loaded: Lib, Main. 
*Main Lib> task188
95962097
\end{lstlisting}


\end{document}
