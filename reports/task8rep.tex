\documentclass[11pt,a4paper]{article}
\usepackage{listings}
\usepackage{hyperref}
\usepackage[utf8x]{inputenc}
\usepackage[russian]{babel}
\usepackage{dirtytalk}
\usepackage{geometry}
\geometry{verbose,a4paper,tmargin=2cm,bmargin=2cm,lmargin=2cm,rmargin=1.5cm}
\title{Project Euler Task 8}
\author{Козиний Сергей}
\begin{document}    
\maketitle
\href{''https://projecteuler.net/problem=8''}{Problem 8}

\say{

The four adjacent digits in the 1000-digit number that have the greatest product are 9 × 9 × 8 × 9 = 5832.\\*
73167176531330624919225119674426574742355349194934\\*
96983520312774506326239578318016984801869478851843\\*
85861560789112949495459501737958331952853208805511\\*
12540698747158523863050715693290963295227443043557\\*
66896648950445244523161731856403098711121722383113\\*
62229893423380308135336276614282806444486645238749\\*
30358907296290491560440772390713810515859307960866\\*
70172427121883998797908792274921901699720888093776\\*
65727333001053367881220235421809751254540594752243\\*
52584907711670556013604839586446706324415722155397\\*
53697817977846174064955149290862569321978468622482\\*
83972241375657056057490261407972968652414535100474\\*
82166370484403199890008895243450658541227588666881\\*
16427171479924442928230863465674813919123162824586\\*
17866458359124566529476545682848912883142607690042\\*
24219022671055626321111109370544217506941658960408\\*
07198403850962455444362981230987879927244284909188\\*
84580156166097919133875499200524063689912560717606\\*
05886116467109405077541002256983155200055935729725\\*
71636269561882670428252483600823257530420752963450\\*

Find the thirteen adjacent digits in the 1000-digit number that have the greatest product. What is the value of this product?

}\\

Для того, чтобы найти максимальное произведение среди всех произведений рядом стоящих цифр в данной последовательности, для начала сгенерируем все подобные произведения:
\begin{lstlisting}[language=Haskell, frame=single]
  --- Haskell lang
  import qualified Data.List as List
  import qualified Data.Char as Char
  numString = "7316717653133062491922511967442657474235534919493496983520312
               7745063262395783180169848018694788518438586156078911294949545
               9501737958331952853208805511125406987471585238630507156932909
               6329522744304355766896648950445244523161731856403098711121722
               3831136222989342338030813533627661428280644448664523874930358
               9072962904915604407723907138105158593079608667017242712188399
               8797908792274921901699720888093776657273330010533678812202354
               2180975125454059475224352584907711670556013604839586446706324
               4157221553975369781797784617406495514929086256932197846862248
               2839722413756570560574902614079729686524145351004748216637048
               4403199890008895243450658541227588666881164271714799244429282
               3086346567481391912316282458617866458359124566529476545682848
               9128831426076900422421902267105562632111110937054421750694165
               8960408071984038509624554443629812309878799272442849091888458
               0156166097919133875499200524063689912560717606058861164671094
               0507754100225698315520005593572972571636269561882670428252483
               600823257530420752963450"
  numDigit = map Char.digitToInt numString
  windowMuls  = List.unfoldr (\lst-> case lst of
                   (x:xs)-> Just (foldl (*) 1 (take 13 lst), drop 1 lst) 
                   []-> Nothing) numDigit
\end{lstlisting}
Здесь: $numString$ --- строковое представление данного числа; $numDigit$ --- представление данного числа в виде последовательности цифр; $windowMuls$ --- подсчитанные произведения по скользящему окну шириной в 13 позиций. Вычисление $windowMuls$ производится при помощи функции $unfoldr$. Процедура \\*
$unfoldr :: (b -> Maybe (a, b)) -> b -> [a] $ (здесь и далее $"::"$ будет означать принадлежность к типу)
по заданному зерну $seed :: b$ воссоздает последовательность элементов $seq :: a$ с помощью функции $func :: b -> Maybe (a, b)$, где на каждом шаге, правый элемент вычисляемой пары, обернутой в монаду $Maybe$ включается в результирующий список  а левый - служит аргументом для $func$ на следующем шаге итерации. Если на некотором шаге $Maybe(a,b)$ принимает значение $Nothing$, список заканчивается и вычисление обрывается. 

В итоге: 
\begin{lstlisting}[language=Haskell, frame=single]
task8::Int
task8 = maximum windowMuls


\end{lstlisting}

Результат:
\begin{lstlisting}frame=single]
#Console
austrotaxus@small-box:~/EulerProj$ stack ghci
Using main module: Package `EulerProj' component exe:EulerProj-exe
with main-is file: /home/austrotaxus/EulerProj/app/Main.hs
The following GHC options are incompatible with GHCi and have not
been passed to it: -threaded
Configuring GHCi with the following packages: EulerProj
GHCi, version 7.10.3: http://www.haskell.org/ghc/  :? for help
[1 of 2] Compiling Lib
( /home/austrotaxus/EulerProj/src/Lib.hs, interpreted )
[2 of 2] Compiling Main
( /home/austrotaxus/EulerProj/app/Main.hs, interpreted )
Ok, modules loaded: Lib, Main. 
*Main Lib> task8
23514624000
\end{lstlisting}


Проект, содержащий это и другие решения можно найти по адрессу:\\*

\url{https://github.com/Austrotaxus/EulerProj/}

\end{document}
