\documentclass[11pt,a4paper]{article}
\usepackage{listings}
\usepackage{hyperref}
\usepackage[utf8x]{inputenc}
\usepackage[russian]{babel}
\usepackage{dirtytalk}
\usepackage{geometry}
\geometry{verbose,a4paper,tmargin=2cm,bmargin=2cm,lmargin=2cm,rmargin=1.5cm}
\title{Project Euler Task 4}
\author{Козиний Сергей}
\begin{document}    
\maketitle
\href{''https://projecteuler.net/problem=4''}{Problem 4}

\say{ 

A palindromic number reads the same both ways. The largest palindrome made from the product of two 2-digit numbers is 9009 = 91 × 99.

Find the largest palindrome made from the product of two 3-digit numbers.
}\\

Для решения задачи реализуем функцию проверки на палиндромность. Для этого число необходимо развернуть в список содержащихся в нем цифр, а затем проверить равен ли такой список самому себе, развёрнутому в обратном направлении:

\begin{lstlisting}[language=Haskell, frame=single]
  --Haskell lang
import qualified Data.List as List
toList x= List.unfoldr (\a -> case a of
                                         0 -> Nothing
                                         _ -> Just (mod a 10,div a 10)) x

isPoly x = x == reverse x
  
\end{lstlisting}
Процедура \\*
$unfoldr :: (b -> Maybe (a, b)) -> b -> [a] $ (здесь и далее $"::"$ будет означать принадлежность к типу)
по заданному зерну $seed :: b$ воссоздает последовательность элементов $seq :: a$ с помощью функции $func :: b -> Maybe (a, b)$, где на каждом шаге, правый элемент вычисляемой пары, обернутой в монаду $Maybe$ включается в результирующий список  а левый - служит аргументом для $func$ на следующем шаге итерации. Если на некотором шаге $Maybe(a,b)$ принимает значение $Nothing$, список заканчивается и вычисление обрывается. 

Теперь построим список всех палиндромов трёхзначных чисел и найдем максимальный из них
\begin{lstlisting}[language=Haskell, frame=single]
  --Haskell lang
task4 = maximum \$ filter (isPoly . toList)[a*b|a<-[999,998..100], b<-[999,998..100]]
    where toList x= List.unfoldr (\a -> case a of
                                         0 -> Nothing
                                         _ -> Just (mod a 10,div a 10)) x
          isPoly x = x == reverse x

\end{lstlisting} 

Результат:


\begin{lstlisting}[frame=single]

#Console
austrotaxus@small-box:~/EulerProj$ stack ghci
Using main module: Package `EulerProj' component exe:EulerProj-exe
with main-is file: /home/austrotaxus/EulerProj/app/Main.hs
The following GHC options are incompatible with GHCi and have not
been passed to it: -threaded
Configuring GHCi with the following packages: EulerProj
GHCi, version 7.10.3: http://www.haskell.org/ghc/  :? for help
[1 of 2] Compiling Lib
( /home/austrotaxus/EulerProj/src/Lib.hs, interpreted )
[2 of 2] Compiling Main
( /home/austrotaxus/EulerProj/app/Main.hs, interpreted )
Ok, modules loaded: Lib, Main. 
*Main Lib> task4
906609

\end{lstlisting}


\end{document}
